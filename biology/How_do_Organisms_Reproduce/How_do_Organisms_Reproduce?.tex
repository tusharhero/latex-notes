\documentclass[A4]{article}
\usepackage{graphicx}
\usepackage{dirtytalk}
\newcommand{\vsauce}{\includegraphics[scale=0.015]{vsauce-eyes.jpg}}
\usepackage{xcolor}
\newcommand{\darkmode}{\pagecolor[rgb]{0.01,0.01,0.01}}
%\darkmode
\newcommand{\whitetext}{\color[rgb]{0.6,0.6,0.6}}
%\whitetext

\graphicspath{ {./images/} }
\title{How do Organisms Reproduce?}
\author{Tushar Maharana}
\begin{document}
    \maketitle
    \tableofcontents
    \section{Why even reproduce?}
    After all, reproduction is not necessary to maintain the life of an individual organisms, unlike Respiration etc.
    And creating a new individual, would take a lot of effort and energy. Why go through this trouble?
    \\
    If a species doesn't reproduce it won't be able to survive and will go extinct. If a species had only one individual and it doesn't reproduce, we would never notice it.
    \\ Wait, I have a question \\
    How do we know that two different individual organisms belong to the same species?
    \vsauce \\
    Usually,we say this because they look similar to each other.
    Thus, reproducing organisms create new individuals that look very much like themselves.
    \section{Do organisms create exact copies of each other?}
    Reproduction at its most basic level will involve making copies of the blueprints of body design. \\
    Chromosomes in the nucleus of a cell contain information for inheritance of features from parents to next generation in the form of DNA (Deoxyribo Nucleic Acid) molecules.\\
    The DNA in the cell nucleus is the information source for making proteins. If DNA(information) is changed, proteins will change. These different proteins will lead to altered body designs.\\
    \paragraph{}
    So the most basic event in reproduction is the creation of a DNA copy. Cells use various chemical reactions to replicate the DNA.\\
    But simply creating this copy and pushing it out won't help, because it won't have the supporting cellular apparatus. \\
    Therefore, DNA copying is accompanied by the creation of an additional cellular apparatus, and then the DNA copies separate, each with its own cellular apparatus. \\
    Effectively, a cell divides to give rise to two cells.
    \paragraph{}
    These two cells are of course similar, but are they likely to be absolutely identical? \vsauce
    The answer to this question will depend on how accurately the copying reactions involved occur. 
    No bio-chemical reaction is absolutely reliable. 
    Therefore, it is only to be expected that the process of copying the DNA will have some variations each time.
    And because of this, the copied DNA is similar but not identical to the original DNA.
    Some of thsese DNA variations maybe so drastic that the organism just dies because of the cellular organelles are not able to work with DNA. 
    \paragraph{}
    Here is the cool part, there maybe some variations which are not drastic and hence these variations stay on and only these types of variations go to the next generation.\\ 
    \paragraph{}
    {\Huge{and this leads to EVOLUTION!}}
    \section{Importance of Variations}
    \paragraph{}
    Populations of organisms fill well-defined places, or niches, in the ecosystem. 
    This means that their body designs are suited for this particular niche. 
    using their ability to reproduce they pass this on to the next generations.
    But if the environment changes or the niche changes, the organisms will be wiped out or in simple terms, go \textbf{EXTINCT}
    \paragraph{}
    Okay, That's cool. But What the duck is a 'niche'? \vsauce \\
    Niche is particular environmental or physical features or factors to which a organism needs to adapt to.
    \paragraph{}
    Let's say,
    You belong to a species of bacteria which lives in the ocean and feeds off of the chemical soup served by the earth.
    Your species is very much used to this. \\ 
    But then it happens... \\
    \say{What happens?} \\
    The temperature rises and all members of your species \emph{die}. But wait, why didn't you die? \\
    Maybe because you have a variation in your DNA. Which let's you survive and reproduce(Asexual ig?). \\
    yay! you just saved your species.\\
    \say{\emph{Variation is thus useful for the survival of species over time.}} \\
    lets say I lied and all of that didn't happen and you die in your teens because of the cold temperature.\\
    \emph{that's dark}\\
    In this case the variation doesn't help you but helps the species as a whole by getting rid of you.
     But if the temperature really did rise, you would survive and this would benefit the species.
     \say{\emph{Variations are not neccesarily good for the individual but definitely useful for the species as a whole.}} 

     \section{Modes of asexual reproductions used by single-celled organisms}
    Let's start with an experiment, Take 100ml of water and add 10 gm of sugar to it. 
    Take 20ml of this solution and add Yeast granules to it. Put a cotton plug on the bottleneck. Keep in warm place and wait.
    After a while make a slide from it. \\ look at it using a Microscope. \\ \\
     \includegraphics[scale=0.25]{Yeast-Under-Microscope.jpg} \\ \\
     Now take a bread pour water on it and then store it in a cool place. \\ \\
     \includegraphics[scale=0.15]{bread-mould-fungus.jpg} \\ \\
     Zoom in on it, and I bet you will see this.  \\
     You might see that it works differently for the two for the Yeast we need a warm place but for the bread mould we need a cool place.
     \paragraph{}
     So far we had discussed why reproduction is important, and why variation is important, in this section we will see how organisms actually reproduce.
     By the way, if you are wondering what 'Asexual' means, it is nothing but 
     \say{A mode of reproduction in which only one parent is involved}
     \subsection*{Fission}
     Fission is just the division of something, i'e cells in this case. 
     For unicellular organisms cell divison is reproduction.
     There are many ways to do this, many bacteria and protozoa just split into two cells.
     \paragraph{}
     In Amoeba, the cell can split itself in any direction (or in fancy terms \emph{plane}).
     \includegraphics[scale=0.5]{Binary-fission-in-amoeba.png}
\end{document}